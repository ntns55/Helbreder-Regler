\chapter*{Niveau 3}
\addcontentsline{toc}{chapter}{Niveau 3}
Når du helbreder ser du livets lys. Du fanger det ikke men guider det tilbage til dem der har brug for det. Du vil altid kunne finde venner når du er i stand til at give liv til hvad der burde være dødt. 
\begin{table}[H]
    \centering
    \begin{tabular}{|p{0.50\textwidth}|p{0.25\textwidth}|}
    \rowcolor{cerulean!80}\hline
        Evne navn & Pris i XP \\\hline
        Ekstra NK Niv. 2 & 1\\\hline
        Kur & 2\\\hline
        Sygdoms samler & 2 \\\hline
        Triage & 2\\\hline
        Urtemidler Niv. 3 & 2 \\\hline
    \end{tabular}
\end{table}
\section*{Evne beskrivelse}
\addcontentsline{toc}{section}{Evne beskrivelse}

\subsection*{Ekstra NK Niv. 2}
\addcontentsline{toc}{subsection}{Ekstra NK Niv. 2}
Du har et ekstra nævekamp.\\

\subsection*{Kur}\addcontentsline{toc}{subsection}{Kur}
Lægen kan kurere Søvn, Paralyse, smerteeffekter, samt Pest og
sygdomme. Dette skal rollespiles og vil tage 10 sekunder.

\subsection*{Sygdoms samler}\addcontentsline{toc}{subsection}{Sygdoms samler}
Når helbrederen bruger evnen kur på en sygdom vil helbrederen kunne skrive denne ned i en bog. De vil være i stand til at kunne genskabe denne sygdom, eller en muteret version af den, ved at snakke med en arrangør.

\input{../Evne-Ordbog/Triage.tex}

\subsection*{Urtemidler Niv. 3}\addcontentsline{toc}{subsection}{Urtemidler Niv. 3}

Du kan bruge urte til at helbreder de syge. Når du laver førstehjælp eller Helbred sår vil du kunne få en effekt afhængig af hvilken urte du bruger. Disse urter vil blive brugt med denne evne og skal afleveres til en arrangør. Det er ikke muligt at bruge mere end en urt per patient.
\begin{table}[H]
     \centering
    \begin{tabular}{|p{0.25\textwidth}|p{0.25\textwidth}|p{0.25\textwidth}|}
    \rowcolor{cerulean!80}\hline
        Urtenavn & Brug & Effekt \\\hline
        Dværgerod & Lav en te på denne urt. & Når du helbreder med denne urt vil din patient få halvdelen af deres mana tilbage. De kan ikke kaste magi de næste 10 minutter.\\\hline
        Hybenholdt &Denne skal sprinkles over såret& Når du bruger helbred sår vil din patient starte på halvt liv, men du \emph{skal} helbrede dem til fuldt liv og må ikke stoppe undervejs med mindre du bliver dræbt.\\\hline
        Cedertræ &Lav en te på denne urt.& Når du bruger Helbred sår vil din patient få fuld liv efter du ville minimum have givet dem 1 liv.\\\hline
    \end{tabular}
\end{table}

 